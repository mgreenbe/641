    \documentclass[12pt]{amsart}

    \usepackage{amsmath, amssymb, amsthm, url}
    \usepackage{mathrsfs, fullpage, amsmath, amssymb, graphicx, xcolor, tikz}

\renewcommand{\phi}{\varphi}
\renewcommand{\epsilon}{\varepsilon}
\renewcommand{\hat}{\widehat}
\renewcommand{\tilde}{\widetilde}
\renewcommand{\bar}{\overline}
\newcommand{\bx}{\boldsymbol{x}}
\newcommand{\by}{\boldsymbol{y}}
\newcommand{\cA}{\mathscr{A}}
\newcommand{\cB}{\mathscr{B}}
\newcommand{\cC}{\mathscr{C}}
\newcommand{\cD}{\mathscr{D}}
\newcommand{\cF}{\mathscr{F}}
\newcommand{\cG}{\mathscr{G}}
\newcommand{\cL}{\mathscr{L}}
\newcommand{\cN}{\mathscr{N}}
\newcommand{\cO}{\mathscr{O}}
\newcommand{\cP}{\mathscr{P}}
\newcommand{\cX}{\mathscr{X}}
\newcommand{\cY}{\mathscr{Y}}
\newcommand{\PP}{\mathbb{P}}
\newcommand{\RR}{\mathbb{R}}
\newcommand{\ZZ}{\mathbb{Z}}
\newcommand{\lra}{\longrightarrow}
\newcommand{\One}{\mathbf{1}}
\newcommand{\blank}{\,\cdot\,}

\newcommand{\vpi}{\boldsymbol{\pi}}
\newcommand{\vtheta}{\boldsymbol{\theta}}
\newcommand{\vzero}{\boldsymbol{0}}
\newcommand{\va}{\boldsymbol{a}}
\newcommand{\vb}{\boldsymbol{b}}
\newcommand{\ve}{\boldsymbol{e}}
\newcommand{\vf}{\boldsymbol{f}}
\newcommand{\vg}{\boldsymbol{g}}
\newcommand{\vu}{\boldsymbol{u}}
\newcommand{\vv}{\boldsymbol{v}}
\newcommand{\vw}{\boldsymbol{w}}
\newcommand{\vx}{\boldsymbol{x}}
\newcommand{\vy}{\boldsymbol{y}}
\newcommand{\vz}{\boldsymbol{z}}
\newcommand{\vU}{\boldsymbol{U}}
\newcommand{\vV}{\boldsymbol{V}}
\newcommand{\vW}{\boldsymbol{W}}
\newcommand{\vX}{\boldsymbol{X}}
\newcommand{\vY}{\boldsymbol{Y}}
\newcommand{\vZ}{\boldsymbol{Z}}

\newcommand{\xhat}{\hat{x}}
\newcommand{\yhat}{\hat{y}}
\newcommand{\bxhat}{\hat{\bx}}
\newcommand{\byhat}{\hat{\by}}
\newcommand{\betahat}{\hat{\beta}}

\newcommand{\iid}{i.i.d.\ }

\newcommand{\simiid}{\overset{\text{\textsc{IID}}}\sim}

\DeclareMathOperator{\expit}{expit}
\DeclareMathOperator{\logit}{logit}
\DeclareMathOperator{\Prob}{Prob}
\DeclareMathOperator{\Bias}{Bias}
\DeclareMathOperator{\EE}{E}
\DeclareMathOperator{\Cov}{Cov}
\DeclareMathOperator{\cov}{cov}
\DeclareMathOperator{\Var}{Var}
\DeclareMathOperator{\Mult}{Mult}
\DeclareMathOperator{\Ber}{Ber}
\DeclareMathOperator{\var}{var}
\DeclareMathOperator{\mean}{mean}
\DeclareMathOperator{\MSE}{MSE}
\DeclareMathOperator{\SSE}{SSE}
\DeclareMathOperator{\ESS}{ESS}
\DeclareMathOperator{\RSS}{RSS}
\DeclareMathOperator{\TSS}{TSS}
\DeclareMathOperator{\argmin}{argmin}
\DeclareMathOperator{\Argmin}{\mathop{\argmin}}


\newtheorem{theorem}{Theorem}
\newtheorem{lemma}[theorem]{Lemma}
\newtheorem{thmdef}[theorem]{Theorem-Definition}
\newtheorem{definition}[theorem]{Definition}
\theoremstyle{remark}
\newtheorem{remark}[theorem]{Remark}
\newtheorem{example}[theorem]{Example}


\setlength\parskip{0.5em}
\setlength\parindent{0em}

\definecolor{orange}{HTML}{FF7F0E}
\definecolor{green}{HTML}{2CA02C}
\definecolor{blue}{HTML}{1F77B4}

\DeclareRobustCommand\orangeline{\raisebox{0.5ex}{\tikz \draw[orange, thick] (1, 0) -- (0.5, 0);}}
\DeclareRobustCommand\blueline{\raisebox{0.5ex}{\tikz \draw[blue, thick] (1, 0) -- (0.5, 0);}}

    \newcommand{\sol}{\bigskip\noindent\textbf{Solution: }}

    \newcommand{\vmu}{\boldsymbol{\mu}}
    \newcommand{\vSigma}{\boldsymbol{\Sigma}}
    \DeclareMathOperator{\Cat}{Categorical}

    \begin{document}
    \title{Assignment 2: Hints}
    \maketitle

        Let $A_i=\nabla^2\ell_i(a, b)$ and let $A=\nabla^2\ell(a, b)=\sum A_i$.

        (1b) Show that $A_i$ has two nonnegative eigenvalues.

        (1c) Find a basic solution of the system
        \[
            A_i\begin{bmatrix}
                s\\t
            \end{bmatrix}=\begin{bmatrix}
                0\\0
            \end{bmatrix}.
        \]

        (1d) An element of $N(A_i)\cap N(A_j)$ is a solution of the homogeneous system
        \[
            \begin{bmatrix}
                A_i\\A_j
            \end{bmatrix}\begin{bmatrix}
                s\\t
            \end{bmatrix}=\begin{bmatrix}
                0\\0
            \end{bmatrix}.
        \]
        Use the fact that $x_i\neq x_j$ to deduce that this system has only the trivial solution.

        (1e) Let $\vx$ be such that $\vx^TA\vx=0$.
        We need to show that $\vx=0$. Note that
        \[
            \vx^T A\vx = \sum\vx^TA_i\vx.
        \]
        The $\vx^TA_i\vx$ are nonnegative (why?), so $\vx^TA\vx=0$ if and only if $\vx^T A_i\vx=0$ for all $i$.
        By the Lemma below, this holds if and only if $\vx\in N(A_i)$ for all $i$...


        \textbf{Lemma:} If $S$ is a positive semidefinite, symmetric matrix, then
        \[
            \{\vy : \vy^TS\vy=0\}=N(S)
        \]

        \textbf{Proof:} Clearly, $N(S)\subseteq\{\vy : \vy^TS\vy=0\}$.
        Conversely, suppose $\vy^TS\vy=0$.
        Since $S$ is symmetric and positive semidefinite, there is a 
        matrix $R$ of real numbers such that $S = R^TR$.
        (This follows from the the Spectral Theorem.) Then
        \[
            0=\vy^TS\vy = \vy^TR^TR\vy = (R\vy)^T(R\vy) = \|R\vy\|^2,
        \]
        so $R\vy$ must be zero. Thus, $\vy\in N(R)\subseteq N(S)$.

        (2a) Use the fact that $0<\sigma(x)<1$ for all $x$.

        (2b) Show that
        \[
            \lim_{t\to\infty}\ell_i(tv_1, tv_2)=\infty
        \]
        if and only if
        \begin{itemize}
            \item $(v_1+v_2x_1)>0$, if $y_i=0$.
            \item $(v_1+v_2x_1)<0$, if $y_i=1$.
        \end{itemize}

        (2c) I'll accept some nicely drawn pictures illustrating what's going on here in lieu of a formal proof. 

        (2d) Suppose $y_i=y_k=0$ and $y_j=1$. By (b),
        \[
            H_i=H(\vw_i),\quad H_j=H(-\vw_j),\quad\text{and}\quad
            H_k=H(\vw_k),
        \]
        where $\vw_i=\begin{bmatrix}
            1&x_i
        \end{bmatrix}^T$.
        Show that $\vw_j = a\vw_i + b\vw_k$ with $a, b>0$. Then invoke (c).

        (2e) Use (b).
    \end{document}