    \documentclass[12pt]{amsart}

    \usepackage{amsmath, amssymb, amsthm, url}
    \usepackage{mathrsfs, fullpage, amsmath, amssymb, graphicx, xcolor, tikz}

\renewcommand{\phi}{\varphi}
\renewcommand{\epsilon}{\varepsilon}
\renewcommand{\hat}{\widehat}
\renewcommand{\tilde}{\widetilde}
\renewcommand{\bar}{\overline}
\newcommand{\bx}{\boldsymbol{x}}
\newcommand{\by}{\boldsymbol{y}}
\newcommand{\cA}{\mathscr{A}}
\newcommand{\cB}{\mathscr{B}}
\newcommand{\cC}{\mathscr{C}}
\newcommand{\cD}{\mathscr{D}}
\newcommand{\cF}{\mathscr{F}}
\newcommand{\cG}{\mathscr{G}}
\newcommand{\cL}{\mathscr{L}}
\newcommand{\cN}{\mathscr{N}}
\newcommand{\cO}{\mathscr{O}}
\newcommand{\cP}{\mathscr{P}}
\newcommand{\cX}{\mathscr{X}}
\newcommand{\cY}{\mathscr{Y}}
\newcommand{\PP}{\mathbb{P}}
\newcommand{\RR}{\mathbb{R}}
\newcommand{\ZZ}{\mathbb{Z}}
\newcommand{\lra}{\longrightarrow}
\newcommand{\One}{\mathbf{1}}
\newcommand{\blank}{\,\cdot\,}

\newcommand{\vpi}{\boldsymbol{\pi}}
\newcommand{\vtheta}{\boldsymbol{\theta}}
\newcommand{\vzero}{\boldsymbol{0}}
\newcommand{\va}{\boldsymbol{a}}
\newcommand{\vb}{\boldsymbol{b}}
\newcommand{\ve}{\boldsymbol{e}}
\newcommand{\vf}{\boldsymbol{f}}
\newcommand{\vg}{\boldsymbol{g}}
\newcommand{\vu}{\boldsymbol{u}}
\newcommand{\vv}{\boldsymbol{v}}
\newcommand{\vw}{\boldsymbol{w}}
\newcommand{\vx}{\boldsymbol{x}}
\newcommand{\vy}{\boldsymbol{y}}
\newcommand{\vz}{\boldsymbol{z}}
\newcommand{\vU}{\boldsymbol{U}}
\newcommand{\vV}{\boldsymbol{V}}
\newcommand{\vW}{\boldsymbol{W}}
\newcommand{\vX}{\boldsymbol{X}}
\newcommand{\vY}{\boldsymbol{Y}}
\newcommand{\vZ}{\boldsymbol{Z}}

\newcommand{\xhat}{\hat{x}}
\newcommand{\yhat}{\hat{y}}
\newcommand{\bxhat}{\hat{\bx}}
\newcommand{\byhat}{\hat{\by}}
\newcommand{\betahat}{\hat{\beta}}

\newcommand{\iid}{i.i.d.\ }

\newcommand{\simiid}{\overset{\text{\textsc{IID}}}\sim}

\DeclareMathOperator{\expit}{expit}
\DeclareMathOperator{\logit}{logit}
\DeclareMathOperator{\Prob}{Prob}
\DeclareMathOperator{\Bias}{Bias}
\DeclareMathOperator{\EE}{E}
\DeclareMathOperator{\Cov}{Cov}
\DeclareMathOperator{\cov}{cov}
\DeclareMathOperator{\Var}{Var}
\DeclareMathOperator{\Mult}{Mult}
\DeclareMathOperator{\Ber}{Ber}
\DeclareMathOperator{\var}{var}
\DeclareMathOperator{\mean}{mean}
\DeclareMathOperator{\MSE}{MSE}
\DeclareMathOperator{\SSE}{SSE}
\DeclareMathOperator{\ESS}{ESS}
\DeclareMathOperator{\RSS}{RSS}
\DeclareMathOperator{\TSS}{TSS}
\DeclareMathOperator{\argmin}{argmin}
\DeclareMathOperator{\Argmin}{\mathop{\argmin}}


\newtheorem{theorem}{Theorem}
\newtheorem{lemma}[theorem]{Lemma}
\newtheorem{thmdef}[theorem]{Theorem-Definition}
\newtheorem{definition}[theorem]{Definition}
\theoremstyle{remark}
\newtheorem{remark}[theorem]{Remark}
\newtheorem{example}[theorem]{Example}


\setlength\parskip{0.5em}
\setlength\parindent{0em}

\definecolor{orange}{HTML}{FF7F0E}
\definecolor{green}{HTML}{2CA02C}
\definecolor{blue}{HTML}{1F77B4}

\DeclareRobustCommand\orangeline{\raisebox{0.5ex}{\tikz \draw[orange, thick] (1, 0) -- (0.5, 0);}}
\DeclareRobustCommand\blueline{\raisebox{0.5ex}{\tikz \draw[blue, thick] (1, 0) -- (0.5, 0);}}

    \newcommand{\sol}{\bigskip\noindent\textbf{Solution: }}

    \newcommand{\vmu}{\boldsymbol{\mu}}
    \newcommand{\vSigma}{\boldsymbol{\Sigma}}
    \DeclareMathOperator{\Cat}{Categorical}

    \begin{document}
    Problem 1:

    a)

    \begin{align*}
        \sigma'(x) &= -(1+e^{-x})^{-2}(-e^{-x})\\
        &= \frac{1}{1+e^{-x}}\frac{e^{-x}}{1 - e^{-x}}\\
        &= \frac{1}{1+e^{-x}}\left(1 - \frac{1}{1 - e^{-x}}\right)\\
        &= \sigma(x)(1-\sigma(x))
    \end{align*}
    Since $0<\sigma(x)<1$ for all $x$, $0<1-\sigma(x)<1$ as well, and $\sigma'(x)>0$.
    b)

    \begin{align*}
    \frac{\partial\ell_i}{\partial a}
    &= -y_i\frac{\sigma'(a+bx_i)}{\sigma(a+bx_i)}
    + (1-y_i)\frac{\sigma'(a+bx_i)}{1 - \sigma(a+bx_i)}\\
    &= -y_i(1-\sigma(a+bx_i)) + (1-y_i)\sigma(a+bx_i)\\
    &= \sigma(a+bx_i) - y_i,\\
    \frac{\partial\ell_i}{\partial b}
    &= -y_i\frac{x_i\sigma'(a+bx_i)}{\sigma(a+bx_i)}
    + (1-y_i)\frac{x_i\sigma'(a+bx_i)}{1 - \sigma(a+bx_i)}\\
    &= -y_ix_i(1-\sigma(a+bx_i)) + (1-y_i)x_i\sigma(a+bx_i)\\
    &= x_i(\sigma(a+bx_i) - y_i),
    \end{align*}
    Therefore,
    \[
        \nabla \ell_i(a, b) = (\sigma(a+bx_i) - y_i)\begin{bmatrix}
            1\\x_i
        \end{bmatrix}.
    \]

    \begin{align*}
        \frac{\partial^2\ell_i}{\partial a^2} &= \sigma'(a+bx_i),\\
        \frac{\partial^2\ell_i}{\partial b^2} &= x_i^2\sigma'(a+bx_i),
        \frac{\partial^2\ell_i}{\partial a\partial b} &= x_i\sigma'(a+bx_i),
    \end{align*}
    Thus,
    \[
        \nabla^2\ell_i(a, b) = \sigma'(a+bx_i) = \begin{bmatrix}
            1&x_i\\x_i&x_i^2.
        \end{bmatrix}
    \]
    The eigenvalues of this matrix are $0$ and $1+x_i^2$, both nonnegative.
    Thus, $\nabla^2\ell_i(a, b)$ is positive semidefinite.


    c)

    The nullspace of $\nabla^2\ell_i(a, b)$ has dimension $1$,
    equal to the multiplicity of 0 as an eigenvalue.
    Therefore, the nonzero vector $\begin{bmatrix}x_i& -1\end{bmatrix}^T$
    is a basis of $N(\nabla^2\ell_i(a, b))$.

    d)

    Since $x_i\neq x_j$, the vectors $\begin{bmatrix}x_i& -1\end{bmatrix}^T$
    and $\begin{bmatrix}x_j& -1\end{bmatrix}^T$ are linearly independent.
    Therefore, the subspaces they span, which, by (c), are the nullspaces of
    $\nabla^2\ell_i(a, b)$ and $\nabla^2\ell_j(a, b)$, respectively, intersect trivially.
    Therefore,
    \[
        \{\vzero\}\subseteq \bigcap_{k=1}^n N(\nabla^2\ell_k(a, b))
        \subseteq N(\nabla^2\ell_i(a, b)) \cap N(\nabla^2\ell_j(a, b))
        = \{\vzero\}.
    \]

    e)

    Suppose $\vx^T\nabla^2\ell(a, b)\vx=0$ for some $\vx\in\RR^2$.
    We must show that $\vx=0$.

    Since $\ell = \sum\ell_i$,
    \[
        \vx^T\nabla^2\ell(a, b)\vx = \sum_{i=1}^n \vx^T\nabla^2\ell_i(a, b)\vx.
    \]
    As $\nabla^2\ell_i(a, b)$ is positive semidefinite for all $i$, by (b),
    $\vx^T\nabla^2\ell_i(a, b)\vx\geq 0$ for all $i$. It follows that
    \[
        \vx^T\nabla^2\ell_i(a, b)\vx = 0
    \]
    for all $i$.

    \textbf{Fact:} If $A$ is a positive-semidefinite, symmetric matrix, there is a
    matrix $B$ of real numbers such that $A = B^TB$. Moreover, $N(A)=N(B)$
    (and $C(A)=C(B^T))$. (This result follows from the Spectral Theorem.)

    Write
    \[
        \nabla^2\ell_i(a, b) = B_i^TB_i.
    \]
    Then
    \[
        0 = \vx^T\nabla^2\ell_i(a, b)\vx =\vx^T B_i^TB_i\vx = (B_i\vx)^T(B_i\vx) = \|B_i\vx\|^2,
    \]
    and it follows that $B_i\vx_i=\vzero$ for all $i$.
    So
    \[
        \vx \in N(B_i)=N(\nabla^2\ell_i(a, b))
    \]
    and, thus,
    \[
        \vx \in \bigcap_{i=1}^nN(\nabla^2\ell_i(a, b)).
    \]
    This intersection is zero by our hypothesis that the $x_i$ are not all equal, and by (d).
    Therefore, $\vx=\vzero$ and $\nabla^2\ell(a, b)$ is positive definite.

    (f)

    Since $\nabla^2\ell(a, b)$ is positive definite, $\ell(a, b)$ is strictly convex.
    Therefore, $\ell(a, b)$ has at most one local minimum.


    Problem 2.

    a)

    Since $0<\sigma(x)<1$ for all $x$, it folloews that $0<p_i(a, b)<1$ for all $a, b$.
    Therefore, $\log p_i(a, b)<0$ and $\ell_i(a, b)=-\log p_i(a, b)>0$.

    b)

    Suppose $$\lim_{t\to\infty}\ell_i(tv_1, tv_2)=\infty.$$
    Then
    \[
        \lim_{t\to\infty}\log p_i(tv_1, tv_1) = -\lim_{t\to\infty}\ell_i(tv_1, tv_2) = -\infty.
    \]
    It follows that
    \[
        \lim_{t\to\infty} p_i(tv_1, tv_2)= 0.
    \]
    Suppose, first, that $y_i=1$. Then
    $$p_i(tv_1, tv_2) = \sigma(t(v_1+v_2x_i))$$
    and
    \[
    \lim_{t\to\infty}\sigma(t(v_1+v_2x_i)) = 0.
    \]
    This happens when $t(v_1+v_2x_i)\to-\infty$ as $t\to\infty$, i.e., when
    \[
        \begin{bmatrix}
            v_1\\v_2
        \end{bmatrix}\cdot\begin{bmatrix}
            1\\x_1
        \end{bmatrix} = v_1+v_2x_i < 0,
    \]
    i.e., when
    \[
    \begin{bmatrix}
        v_1\\v_2
    \end{bmatrix}
    \in H\left(-
    \begin{bmatrix}
    1\\x_i
    \end{bmatrix}
    \right).
    \]

    Suppose, now, that $y_i=0$, so that $p_i(tv_1, tv_2) = 1-\sigma(t(v_1+v_2x_i))$.
    We have
    \[
        \lim_{t\to\infty}(1-\sigma(t(v_1+v_2x_i))) = 0
    \]
    if and only if
    \[
        \lim_{t\to\infty}\sigma(t(v_1+v_2x_i)) = 1
    \]
    if and only if $t(v_1+v_2x_i)\to +\infty$ as $t\to\infty$, i.e., when
    \[
        \begin{bmatrix}
            v_1\\v_2
        \end{bmatrix}\cdot\begin{bmatrix}
            1\\x_1
        \end{bmatrix} = v_1+v_2x_i > 0,
    \]
    i.e., when
    \[
    \begin{bmatrix}
        v_1\\v_2
    \end{bmatrix}
    \in H\left(
    \begin{bmatrix}
    1\\x_i
    \end{bmatrix}
    \right).
    \]
    Setting
    \[
    H_i = \begin{cases}H(\vw_i) &\text{if $y_i=0$},\\
        H(-\vw_i) &\text{if $y_i=1$}
    \end{cases},\quad\text{where}\quad
    \vw_i = \begin{bmatrix}
        1\\x_i
    \end{bmatrix}.
    \]

    d)

            Suppose $y_i=y_k=0$ and $y_j=1$.
            Then
            \[
                H_i=H(\vw_i),\quad H_j=H(-\vw_j),\quad\text{and}\quad
                H_k=H(\vw_k)
            \]
            Note that
            \[
                -(-\vw_j) = \vw_j = a\vw_i + b\vw_k,\quad\text{where}\quad
                a=\frac{x_k-x_j}{x_k-x_i}\quad\text{and}\quad
                a=\frac{x_j-x_i}{x_k-x_i}.
            \]
            Using the inequalities $x_1<x_2<x_3$, one deduces that $0<a, b<1$.
            Therefore, by (c),
            \[
                H_i\cup H_j\cup H_k = \RR^2 - \{\vzero\}.
            \]

            The case where $y_i=y_k=1$ and $y_j=0$ is argued similarly.

        e)

        Let $\vv\in\RR^2$ be nonzero.
        Then $\vv$ belongs to one of $H_i$, $H_j$, or $H_k$.
        Without loss of generality, suppose $\vv\in H_i$. Then, by (b),
        \[
            \lim_{t\to\infty}\ell_i(tv_1, tv_2) = \infty.
        \]
        Since each $\ell(a, b)\geq \ell_i(a, b)$ by (a), we have
        \[
            \lim_{t\to\infty}\ell_i(tv_1, tv_2) = \infty,
        \]
        too. Therefore, that $S_K$ does not contain the ray $\{t\vv:t\geq 0\}$.
        Since $\vv$ was arbitrary, $S_K$ doesn't contain any ray of this form.

        f)

        By Problem 1(b), $\ell(a, b)$ is convex. Therefore $S_K$ is convex.
        Since $S_K$ is convex and, by e), contains no ray, $S_K$ is bounded.
        Being closed and bounded, it's compact.

        g)

        Since $\ell$ is continuous and $S_K$ is compadt, $\ell$ achieves
        achieves its infimum on $S_K$ at $(\hat a, \hat b)\in S_K$, say.
        If $\ell(a,b)\leq \ell(\hat a, \hat b)$,
        then $\ell(a, b)\leq K$ and $(a, b)\in K$. It follows that
        $\ell(a, b)=\ell(\hat a, \hat b)$. Therefore,
        \[
            \ell(\hat a, \hat b) = \inf_{(a,b)\in\RR^2}\ell(a, b).
        \]
        We've assumed our $x_i$ values are all distinct,
        so the hypothesis of 1(f) holds.
        Therefore, $(\hat a, \hat b)$ is the unique point at which
        $\ell$ achieves its minimum value.


        Let $A_i=\nabla^2\ell_i(a, b)$ and let $A=\nabla^2\ell(a, b)=\sum A_i$.

        (1b) Show that $A_i$ has two nonnegative eigenvalues.

        (1c) Find a basic solution of the system
        \[
            A_i\begin{bmatrix}
                s\\t
            \end{bmatrix}=\begin{bmatrix}
                0\\0
            \end{bmatrix}.
        \]

        (1d) An element of $N(A_i)\cap N(A_j)$ is a solution of the homogeneous system
        \[
            \begin{bmatrix}
                A_i\\A_j
            \end{bmatrix}\begin{bmatrix}
                s\\t
            \end{bmatrix}=\begin{bmatrix}
                0\\0
            \end{bmatrix}.
        \]
        Use the fact that $x_i\neq x_j$ to deduce that this system has only the trivial solution.

        (1e) Let $\vx$ be such that $\vx^TA\vx=0$.
        We need to show that $\vx=0$. Note that
        \[
            \vx^T A\vx = \sum\vx^TA_i\vx.
        \]
        The $\vx^TA_i\vx$ are nonnegative (why?), so $\vx^TA\vx=0$ if and only if $\vx^T A_i\vx=0$ for all $i$.
        By the Lemma below, this holds if and only if $\vx\in N(A_i)$ for all $i$...


        \textbf{Lemma:} If $S$ is a positive semidefinite, symmetric matrix, then
        \[
            \{\vy : \vy^TS\vy=0\}=N(S)
        \]

        \textbf{Proof:} Clearly, $N(S)\subseteq\{\vy : \vy^TS\vy=0\}$.
        Conversely, suppose $\vy^TS\vy=0$.
        Since $S$ is symmetric and positive semidefinite, there is a 
        matrix $R$ of real numbers such that $S = R^TR$.
        (This follows from the the Spectral Theorem.) Then
        \[
            0=\vy^TS\vy = \vy^TR^TR\vy = (R\vy)^T(R\vy) = \|R\vy\|^2,
        \]
        so $R\vy$ must be zero. Thus, $\vy\in N(R)\subseteq N(S)$.

        (2a) Use the fact that $0<\sigma(x)<1$ for all $x$.

        (2b) Show that
        \[
            \lim_{t\to\infty}\ell_i(tv_1, tv_2)=\infty
        \]
        if and only if
        \begin{itemize}
            \item $(v_1+v_2x_1)>0$, if $y_i=0$.
            \item $(v_1+v_2x_1)<0$, if $y_i=1$.
        \end{itemize}

        (2c) I'll accept some nicely drawn pictures illustrating what's going on here in lieu of a formal proof. 

        (2d) Suppose $y_i=y_k=0$ and $y_j=1$. By (b),
        \[
            H_i=H(\vw_i),\quad H_j=H(-\vw_j),\quad\text{and}\quad
            H_k=H(\vw_k),
        \]
        where $\vw_i=\begin{bmatrix}
            1&x_i
        \end{bmatrix}^T$.
        Show that $\vw_j = a\vw_i + b\vw_k$ with $a, b>0$. Then invoke (c).

        2(e) Use (b).
    \end{document}