\documentclass[12pt]{amsart}

\usepackage{mathrsfs, fullpage, amsmath, amssymb, graphicx, xcolor, tikz}

\usepackage{mathrsfs, fullpage, amsmath, amssymb, graphicx, xcolor, tikz}

\renewcommand{\phi}{\varphi}
\renewcommand{\epsilon}{\varepsilon}
\renewcommand{\hat}{\widehat}
\renewcommand{\tilde}{\widetilde}
\renewcommand{\bar}{\overline}
\newcommand{\bx}{\boldsymbol{x}}
\newcommand{\by}{\boldsymbol{y}}
\newcommand{\cA}{\mathscr{A}}
\newcommand{\cB}{\mathscr{B}}
\newcommand{\cC}{\mathscr{C}}
\newcommand{\cD}{\mathscr{D}}
\newcommand{\cF}{\mathscr{F}}
\newcommand{\cG}{\mathscr{G}}
\newcommand{\cL}{\mathscr{L}}
\newcommand{\cN}{\mathscr{N}}
\newcommand{\cO}{\mathscr{O}}
\newcommand{\cP}{\mathscr{P}}
\newcommand{\cX}{\mathscr{X}}
\newcommand{\cY}{\mathscr{Y}}
\newcommand{\PP}{\mathbb{P}}
\newcommand{\RR}{\mathbb{R}}
\newcommand{\ZZ}{\mathbb{Z}}
\newcommand{\lra}{\longrightarrow}
\newcommand{\One}{\mathbf{1}}
\newcommand{\blank}{\,\cdot\,}

\newcommand{\vpi}{\boldsymbol{\pi}}
\newcommand{\vtheta}{\boldsymbol{\theta}}
\newcommand{\vzero}{\boldsymbol{0}}
\newcommand{\va}{\boldsymbol{a}}
\newcommand{\vb}{\boldsymbol{b}}
\newcommand{\ve}{\boldsymbol{e}}
\newcommand{\vf}{\boldsymbol{f}}
\newcommand{\vg}{\boldsymbol{g}}
\newcommand{\vu}{\boldsymbol{u}}
\newcommand{\vv}{\boldsymbol{v}}
\newcommand{\vw}{\boldsymbol{w}}
\newcommand{\vx}{\boldsymbol{x}}
\newcommand{\vy}{\boldsymbol{y}}
\newcommand{\vz}{\boldsymbol{z}}
\newcommand{\vU}{\boldsymbol{U}}
\newcommand{\vV}{\boldsymbol{V}}
\newcommand{\vW}{\boldsymbol{W}}
\newcommand{\vX}{\boldsymbol{X}}
\newcommand{\vY}{\boldsymbol{Y}}
\newcommand{\vZ}{\boldsymbol{Z}}

\newcommand{\xhat}{\hat{x}}
\newcommand{\yhat}{\hat{y}}
\newcommand{\bxhat}{\hat{\bx}}
\newcommand{\byhat}{\hat{\by}}
\newcommand{\betahat}{\hat{\beta}}

\newcommand{\iid}{i.i.d.\ }

\newcommand{\simiid}{\overset{\text{\textsc{IID}}}\sim}

\DeclareMathOperator{\expit}{expit}
\DeclareMathOperator{\logit}{logit}
\DeclareMathOperator{\Prob}{Prob}
\DeclareMathOperator{\Bias}{Bias}
\DeclareMathOperator{\EE}{E}
\DeclareMathOperator{\Cov}{Cov}
\DeclareMathOperator{\cov}{cov}
\DeclareMathOperator{\Var}{Var}
\DeclareMathOperator{\Mult}{Mult}
\DeclareMathOperator{\Ber}{Ber}
\DeclareMathOperator{\var}{var}
\DeclareMathOperator{\mean}{mean}
\DeclareMathOperator{\MSE}{MSE}
\DeclareMathOperator{\SSE}{SSE}
\DeclareMathOperator{\ESS}{ESS}
\DeclareMathOperator{\RSS}{RSS}
\DeclareMathOperator{\TSS}{TSS}
\DeclareMathOperator{\argmin}{argmin}
\DeclareMathOperator{\Argmin}{\mathop{\argmin}}


\newtheorem{theorem}{Theorem}
\newtheorem{lemma}[theorem]{Lemma}
\newtheorem{thmdef}[theorem]{Theorem-Definition}
\newtheorem{definition}[theorem]{Definition}
\theoremstyle{remark}
\newtheorem{remark}[theorem]{Remark}
\newtheorem{example}[theorem]{Example}


\setlength\parskip{0.5em}
\setlength\parindent{0em}

\definecolor{orange}{HTML}{FF7F0E}
\definecolor{green}{HTML}{2CA02C}
\definecolor{blue}{HTML}{1F77B4}

\DeclareRobustCommand\orangeline{\raisebox{0.5ex}{\tikz \draw[orange, thick] (1, 0) -- (0.5, 0);}}
\DeclareRobustCommand\blueline{\raisebox{0.5ex}{\tikz \draw[blue, thick] (1, 0) -- (0.5, 0);}}


\begin{document}
\title{Regularization}
\maketitle

\section{Multiple linear regression}
Convention: We view $\RR^k$ as a subset of $\RR^{k+1}$ via the following identification
\begin{equation}\label{E:identification}
    v\in\RR^k\quad\longleftrightarrow\quad (1, v)\in\RR^{k+1}.
\end{equation}
    
$p-1$ predictor variables:
\[
(x_1, y_1),\ldots,(x_n, y_n)\in \RR^{1\times (p-1)}\times \RR
\]

Viewing $x_i$ as an element of $\RR^{1\times p}$ via~\eqref{E:identification}, define:
\[
    x := \begin{bmatrix}x_1\\\vdots\\x_n\end{bmatrix}\in\RR^{n\times p},
    \quad y := \begin{bmatrix}
        y_1\\\vdots\\y_n
    \end{bmatrix}\in\RR^{n\times 1}
\]
For $\beta\in\RR^{p\times 1}$, consider the equation:
\[
    x\beta = y
\]
Equivalently:
\[
    \beta_0 + \beta_1x_{i,1} + \cdots \beta_{p-1}x_{i,p-1} = y_i,\quad i=1,\ldots,n.
\]

\textbf{Recall:}
The \emph{column space of $x$} is the subspace $C(x)$ of $\RR^{n\times 1}$
characterized by any of the following equivalent conditions:
\begin{itemize}\setlength\itemsep{0.5em}
    \item $C(x)$ is the set of all linear combinations of the columns of $x$
    \item $C(x)=\{x\beta : \beta\in\RR^{p\times 1}\}$
    \item $C(x)=\{y\in\RR^{n\times 1} : \text{$x\beta=y$ has a solution}\}$
\end{itemize}
$C(x)$ is also called the \emph{image of $x$}.

Let $\hat y\in\RR^{n\times 1}$ be the vector characterized by any of the
equivalent conditions:
\begin{itemize}\setlength\itemsep{0.5em}
    \item $\displaystyle{\hat y = \argmin\limits_{z\in C(x)}\|z - y\|}$
    \item $\hat y$ is the vector in the column space of $x$ closest to $y$.
    \item $\hat y$ is the orthogonal projection of $y$ onto the column space of $x$.
\end{itemize}
In partricular, $x\beta = \hat y$ has a solution.

\section{The case $\rank(x)=p$}
Suppose $\rank(x)=p$. Then $\beta\mapsto x\beta$ maps $\RR^{p\times 1}$
bijectively onto $C(x)$ and, therefore, there is a unique vector $\hat\beta\in\RR^{p\times 1}$
--- the \emph{least squares solution of} $x\beta=y$ --- such that
\[
    x\hat\beta = \hat y.
\]
The vector $\hat\beta$ is characterized by the fact that it minimizes
the sum of squared errors in approximating $y$ by a vector of the form $x\beta$:
\[
    \hat \beta = \argmin_{\beta \in \RR^{p\times 1}}\|x\beta - y\|^2
\]

Since $\rank(x)=p$, the matrix $x^Tx\in\RR^{p\times p}$ is invertible and the system
\[
    x^Tx\beta = x^Ty
\]
has unique solution; this solution is just $\hat\beta$:
\[
    \hat \beta = (x^Tx)^{-1}x^Ty
\]
Thus,
\[
\hat y = x\hat \beta = Py,
\]
where
\[
    P := x(x^Tx)^{-1}x^T.
\]
The matrix $P$ is called the \emph{projection matrix} because it describes
orthogonal projection from $\RR^{n\times 1}$ onto $C(x)$.

If we view the $y_i$ as realizations of random variable $Y_i$ and let
\[
    Y = \begin{bmatrix}
        Y_1\\\vdots\\Y_n,
    \end{bmatrix}
\]
then we may view 
\[
    \hat\beta = \hat\beta(Y_1,\ldots,Y_n) = (x^Tx)^{-1}x^T Y
\]
as an estimator.

\begin{theorem}
    Suppose $\rank(x)=p$ and
    \[
        Y\sim N(x\beta, \Sigma).
    \]
    Then $\hat\beta$ is an unbiased estimator of $\beta$.
\end{theorem}
\begin{proof} Use the linearity of expectation:
    \begin{multline*}
        \qquad\qquad\EE\hat \beta 
        = \EE\left[(x^Tx)^{-1}x^TY\right] 
        = (x^Tx)^{-1}x^T\EE Y
        \\= (x^Tx)^{-1}x^T (x\beta) 
        = (x^Tx)^{-1}(x^T x)\beta 
        = I\beta 
        = \beta\qquad\qquad\qedhere
    \end{multline*}        
\end{proof}

\begin{align*}
    \Var\hat\beta &= \Var (x^Tx)^{-1}x^TY\\
    &= (x^Tx)^{-1}x^T(\Var Y)((x^Tx)^{-1}x^T)^T\\
    &= (x^Tx)^{-1}x^T(\sigma^2I)x(x^Tx)^{-1}\\
    &= \sigma^2(x^Tx)^{-1}(x^Tx)(x^Tx)^{-1}\\
    &= \sigma^2(x^Tx)^{-1}
\end{align*}



\section{The case $\rank(x)\leq p$}

We consider a \emph{regularized} version of multiple linear regression.
Let $\lambda > 0$ and consider the problem of minimizing 
\[
    \SSE_\lambda(\beta) := \|x\beta - y\|^2 + \lambda^2\|\beta\|^2
\]
Let
\[
    \xi\ := \begin{bmatrix}
        x\\\lambda I^{p\times p}
    \end{bmatrix}\in\RR^{(n + p)\times p},
    \quad
    \eta := \begin{bmatrix}
        y\\0^{p\times 1}
    \end{bmatrix}\in\RR^{(n+p)\times 1}
\]
and consider the equation
\[
    \xi\beta = \eta.
\]
The columns of $\xi$ are linearly independent (why?),
so $\rank(\xi)=p$.
Therefore, by the discussion of the previous section,
$\xi^T\xi$ is invertible and 
\[
    \hat{\beta}_\lambda := (\xi^T\xi)^{-1}\xi^T\eta
\]
minimizes
\[
    \|\xi\beta - \eta\|^2 = \left\|\begin{bmatrix}
        x\beta - y\\\lambda\beta
    \end{bmatrix}\right\|^2 = \|x\beta - y\|^2 + \lambda^2\|\beta\|^2
    = \SSE_\lambda(\beta).
\]

We have:
\begin{align*}
    \xi^T\xi &= \begin{bmatrix}
        x^T&&\lambda I
    \end{bmatrix}\begin{bmatrix}
        x\\\lambda I
    \end{bmatrix}\\
    &= x^Tx + \lambda^2I,\\
    \xi^T\eta &= \begin{bmatrix}
        x^T&&\lambda I
    \end{bmatrix}
    \begin{bmatrix}
        y\\0^{p\times 1}
    \end{bmatrix}\\
    &= x^Ty
\end{align*}

Therefore,
\[
    \hat\beta_\lambda = (x^Tx + \lambda^2I)^{-1}x^Ty.
\]

Let
\[
    W_\lambda = (x^Tx + \lambda^2I)^{-1}x^Tx.
\]

\begin{theorem}
    Suppose $\rank(x)=p$, so that $\hat\beta$ is defined. Then
    \[
        \beta_\lambda = W_\lambda\hat\beta.
    \]
\end{theorem}

\begin{proof} Just compute.
    \begin{align*}
        W_\lambda\hat\beta
        &= (x^Tx + \lambda^2I)^{-1}x^Tx\hat\beta\\
        &= (x^Tx + \lambda^2I)^{-1}x^Tx(x^Tx)^{-1}x^T y\\
        &= (x^Tx + \lambda^2I)^{-1}x^T y\\
        &= \hat\beta_\lambda.\qedhere
    \end{align*}
\end{proof}


\end{document}